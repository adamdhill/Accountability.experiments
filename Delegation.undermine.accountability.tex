\documentclass{article}
\usepackage[comma, authoryear]{natbib}
%\usepackage[msword, basic]{wordlike}

\usepackage{pa}
\usepackage{graphicx}

%\usepackage[normalem]{ulem}

% One inch margins
%\PassOptionsToPackage{margin=1in}{geometry}

% Double spacing
\usepackage{setspace}
\setstretch{1.5}

% Don't justify along the right margin
% \usepackage{ragged2e}
% \RaggedRight

% Format section titles
%\usepackage[uppercase]{titlesec}
%\titlespacing\section{0pt}{0pt}{7pt}
%\usepackage{titlesec}
%\titleformat{\section}
%    {\normalfont\bfseries\filcenter}{\uline{\thesection.\ }}{0em}{\uline}

% Format paragraphs
%\parskip 0pt
%\setlength{\parindent}{0.5in}

% Remove section numbers
%\setcounter{secnumdepth}{-2}



\begin{document}

\title{Does Delegation Undermine Accountability? Experimental Evidence on the Relationship between Blame-Shifting and Control}

\author{Adam Hill\thanks{For generous financial support, the author thanks the UC-Berkeley Experimental Social Sciences Lab and the Berkeley Empirical Legal Studies Fellowship.}\\
UC-Berkeley\\
 \texttt{adh@berkeley.edu}
}

\maketitle

\newpage

\begin{abstract}

A small but growing literature in experimental economics finds that principals can shift responsibility for blameworthy behavior to agents, even when those agents are effectively powerless. Prior work in this field measures blameworthy behavior only indirectly, however. It uses modified dictator games to measure attributions of blame for inequitable allocations of wealth. Yet participants might find inequitable allocations of wealth not blameworthy. Thus, such indirect measures leave open the possibility that prior work is not measuring blame shifting at all. This paper corrects for a crucial shortcoming by providing a direct measure of blame shifting behavior. It reports and discusses first of its kind experimental evidence that shows that principals can delegate to powerless intermediaries in order to evade blame. 

\end{abstract}

\newpage

\section{Introduction}

Can principals delegate the authority to take immoral or anti-social actions to their agents in order to evade responsibility for the consequences of the decision? A burgeoning line of research suggests that principals can shift responsibility to their agents, even where the principal exerts significant control over the agent's decision.\citep{Hamman2010, Coffman2011, Bartling2012} One of the main results, for example, shows that in the context of a dictator game, principals face less punishment by delegating a decision that produces inequitable payoffs to an agent, even if the agent is powerless to choose the equitable outcome. \citep{Grossman2011} Thus, on the basis of this literature it appears that principals can, in fact, employ agents to evade responsibility. 

However, these dictator games measure immoral or anti-social behavior only \emph{indirectly}. These studies interpret inequitable allocations in a dictator as immoral or anti-social behavior, and they interpret third-party decisions to deduct money from the agents' allocations as ``punishment''. \citep{Erat2013} Both of these methodological decisions pose difficulties. These studies provide no evidence that participants in fact construe inequitable payouts as immoral or anti-social. Indeed, Sanjiv Erat suggests that inequitable payouts might be straightforwardly pro-social.\footnote{Erat's own study sought to solve this problem of measuring immoral or anti-social behavior by shifting from a dictator game to a sender-agent-receiver game. In Erat's study, the principal engages in deception. This approach does not solve the problem, however, since deception might well be warranted and in any event we receive no direct evidence of what the participants make of the deception under the circumstances. Erat acknowledges as much: ``Whether or not deception is immoral is admittedly a debatable, and much debated, point.''\citep[p. 274]{Erat2013} Yet despite acknowledging that whether deception is immoral is much debated, Erat notes that his paper ``assume[s] that people do perceive deception to be an immoral act.'' \emph{Id.}} While not all of the studies make available their instructions to participants, those that do make the instructions available do not characterize inequitable payouts as immoral or anti-social. While researchers versed in, for example, public goods literature might be strongly inclined to interpret inequitable results as blameworthy or anti-social,\citep{Hamman2011} one would need further evidence to conclude that the players themselves take inequitable payouts as blameworthy.  In short, the experimental economic games that have thus far driven this burgeoning literature have failed to provide direct measures of players beliefs that inequitable dictator game distributions are in fact blameworthy. 

This paper addresses this shortcoming in the literature. It provides the first study of delegation and blame to \emph{directly} test third-party observers' beliefs about the blameworthiness of the principals' and agents' behaviors. Instead of relying on indirect measures of the immorality or anti-social nature of certain behaviors, the experiments reported here directly measure third-party observers' evaluations of the blameworthiness of the principals' and agents' behavior. Using vignettes, the experiments reported here operationalize the behavior of principals and agents in the context of legislative delegation to administrative agencies. After reading the vignettes, participants in the study rated the blameworthiness of the actors in several treatments, including a treatment in which the principal delegates to a powerless intermediary agent. Even in these cases, where the agent is effectively powerless to change the outcome, participants blame principals significantly less than in cases where the principal brings about the outcome directly. Accordingly, this paper provides the first clean evidence that principals can delegate to a blameworthy decision to a powerless intermediary and nonetheless evade responsibility.

\subsection{Legislative Delegation of Policymaking Authority}

This paper studies the relationship between delegation and blame in the context of legislative delegation of policymaking authority to administrative agencies. In such situations, we are often confronted with the question of how voters allocate blame when Congress and an agency are jointly involved in creating unpopular policy. Ideally rational voters with perfect information would attribute blame in proportion to the relative responsibility of each institution. \citep{Cutler2004} The complexity of contemporary policymaking, however, makes certain types of information material to the process of blame attribution highly costly to obtain. \citep{Gersen2010b} Across a wide range of policy areas, including financial regulation \citep{Romano2012}, the environment \citep{Schoenbrod1993}, telecommunications \citep{Arnold1990}, labor \citep{Farber1991}, and international trade \citep{Epstein1999}, information concerning the relative responsibility of Congress and the agencies in producing policy is effectively unobservable to voters. How do voters allocate blame for unpopular policies in the face of such uncertainty? This paper reports experimental evidence designed to help answer this question. 

One prominent line of thought, the Clarity Thesis, argues that voters tend to resolve such uncertainty in favor of legislatures. \citep{Fox2011, Majone1999, Fiorina1982, Nzelibe2010, Gersen2010b, Mesquita.draft} Where voters know that an agency was involved in the policymaking process, but do not know the extent of its role, the Clarity Thesis predicts that they will under-assign blame to legislatures and over-assign blame to agencies.\footnote{This tendency is not unreasonable. From a principal-agent perspective, voters are the ultimate principals in a democracy. Elected representatives are their agents. \citep{Weingast1984} Administrative agencies are voters' sub-agents. As a large literature on the social psychology of trust and cooperation shows, individuals tend to be more trusting of those with whom they have closer, more frequent interactions. \citep{Tyler2010} While a large gulf exists between voters and their elected representatives, voters might be more trusting of their representatives, with whom they have close, more frequent contact, and thus resolve factual uncertainty in their favor by shifting blame to agencies. Further, voters might be more trusting of their elected representatives because they exercise more control over them. \citep{Ferejohn1999} Nonetheless, legislatures can leverage this dynamic in order to escape blame for unpopular policies to agencies.} Legislatures that exploit this tendency can pass unpopular polices yet escape blame. 

The Clarity Thesis thus highlights one of the core problems voters face as they seek to hold officials accountable: How to apportion comparative blame in the face of incomplete information? Delegation increases ``the informational burden on voters,'' \citep[p. 623]{Nzelibe2010b} thus raising the costs of accurately assigning blame to officials for policy failures. Inaccurate assessments of blame for unpopular policies, in turn, weakens the sanctioning function of retrospective voting \citep{Ferejohn1986}. Raising the costs of electoral discipline effectively ``insulates [legislators] from political retribution''  \citep[p. 101]{Arnold1990} and over-punishes agencies. Broad delegations of authority --- those, for instance, which do not establish detailed standards for administrative action --- are thought to offer the most insulation to legislators.\citep{Fiorina1982} Broad delegations can weaken or break the ``traceability chain,'' linking political actors to specific policy choices. \citep{Arnold1990} While delegation allows legislators to shirk, voters \emph{shift} their blame to agencies, thereby forcing them to take the political brunt of unpopular decisions. The net effect is that shirk-prone representatives can exploit these tendencies and pursue sub-optimal policies with greater frequency.\footnote{Because this dynamic is driven by the informational burdens on voters created by delegation, it is not limited to cases of legislative delegation to traditional Executive Branch agencies; rather, it applies equally well to delegations of policymaking authority to independent agencies and private actors \citep{Rodriguez2010}.} \citep[p. 841]{Fox2011} On this view, limiting legislators' ability to ``disguise their responsibility,'' \citep[p. 47]{Fiorina1982} requires the creation of institutional arrangements that allow voters to ``observe the decisions made by each institution.'' \citep[p. 145]{Nzelibe2010b} Only then will there be ``no `clarity of responsibility' problem.'' (\emph{Id}.)  

In order to test how delegation affects voter performance at assigning blame, this paper reports the findings of three rounds of experiments. The first round of experiments reported here tests the claim that delegation \emph{indirectly} affects voter behavior by raising the costs of information. Experiments 1a and 1b test whether the observability of certain types of information impacts blame attributions. These experiments provide evidence of whether voters tend to allocate more blame to legislatures as the quantity of observed information increases. The second round tests whether delegation \emph{directly} affects voter behavior. The Clarity Thesis predicts that, if voters possess complete information as to the respective contributions of the legislature and agency, they will accurately apportion blame. A burgeoning experimental literature carried out by experimental economists and moral psychologists, however, has found that delegation reduces the blame principals face even in cases where participants possess complete information. \citep{Hamman2010, Coffman2011} If that same dynamic applies in the context of politics, the Clarity Thesis' claim that delegation affects blame exclusively through information would need revisiting. Where the first two rounds of experiments test the blame faced by \emph{legislatures}, the third round of experiments tests how delegation affects the blame faced by \emph{agencies}. 

The results presented here tend to support the Clarity Thesis' claim that, in the absence of full information, voters will tend to shift blame away from legislatures. Information asymmetries indirectly affect blame. Strikingly, however, participants tend to under-assign blame to legislatures even where they are presented with a complete description of the respective actions of the legislature and agency. In other words, participants' tendency to blame legislatures does \emph{not} seem to depend entirely on voters being ``imperfectly informed'' \citep[p. 1044]{Stephenson2006} of the respective contributions of the legislature and agency, as the Clarity Thesis argues. Delegation appears to impact voter behavior directly, as well as indirectly.
 
While delegation tends to reduce blame for legislators, its effects on agencies are more variable. Broadly speaking, attributions of agency blame rationally reflect comparative differences in the power and autonomy of agencies. Attributions of agency blame are sensitive to the degree of legislative oversight and control over the agency and the type of policymaking authority that agencies exercise. The less control the legislature exercises over the agency, and the more responsibility for creating the legal norm the agencies possess, the more blame they face. For instance, an agency exercising more expansive authority --- exercising the power, that is, to create law or policy --- will face more blame than an agency exercising more narrow authority, such as the authority to merely enforce norms created directly by a legislature. While participants' \emph{comparative} attributions of blame matched the responsibility exercised by the agency, less responsible agencies tend to face more blame than is rationally warranted (that is, than is predicted by participants' attributions of blame in the case of direct, non-delegated action), while more responsible agencies reliably face \emph{less} blame than is rationally warranted. While constrained agencies face less blame than autonomous agencies, constrained agencies face more blame than is rationally warranted, and autonomous agencies less blame than is rationally warranted. This result suggests that voters might hedge their bets as their knowledge of the secondary or sub-agents diminishes.

These results have ramifications for foundational administrative law doctrines. Post-\emph{Chevron} approaches to accountability have sought, in varying degrees, to give ``maximum authority to the most politically responsive decision maker,'' on grounds that doing so ``maximizes the responsiveness of policy to majoritarian preferences.'' \citep[p. 56]{Stephenson2008} Agency heads, of course, are unelected. If, however, some elected office that possesses leverage over agency behavior can be reliably held to account for agency performance, voters might be able to hold agencies \emph{derivatively} accountable. Agency policy choices that draw the ire of the electorate, on this view, can be expressed through punishment of the President, who is directly accountable to voters. With the demise of other modes of administrative accountability --- interest group pluralism, for instance \citep{Bressman2003} --- it is derivative accountability that animates approaches to administrative accountability. The findings reported here, however, suggest that derivative electoral accountability is subject to bias caused by the dynamics of blame associated with delegation. If derivative accountability is thought to operate via the Executive Branch --- that is, if citizens are to blame the President for agency action --- the public's evaluation of the policies of agencies may tend, in the case of less powerful agencies, to be inflated, and, in the case of more powerful agencies, to be deflated. In other words, since delegation tends to clump attributions of agency blame, which will result in relatively blameless agencies facing undue heat from the public, and relatively blameworthy agencies getting a pass. If derivative liability is thought to run through the legislature, the problems are even more straightforward: delegation reduces the blame the principal faces. Delegation tends to artificially inflates the public's opinion of  legislative action, relative to a baseline of direct legislative action. Such skewed valuations of agency action sits uneasily alongside the post-\emph{Chevron} ``commitment[] to electoral accountability.'' \citep[p. 634]{Manning1996} If, however, we identify the way in which attributions of blame are sensitive to the legal structure governing relationship between legislature and agency and the type of policymaking authority exercised by the agency, we can begin to consider institutional designs that better accomplish the twin goals of robust agency policymaking and democratic accountability. Most optimistically, we can assess the tendencies that push observers to blame more in one direction or another and institutionally harness these tendencies to enhance accountability.

The paper is organized as follows. Section II sets out the fundamental features of the Clarity Thesis. Sections III, IV, and V explain the design of the three experiments reported here, including how they operationalized the concepts discussed in Section II, and discusses the data. Finally, Section VI addresses the policy implications of the results and broaches avenues for future research. 

\section{The Clarity Thesis: The Difficulty of Allocating Responsibility in the Administrative State}

The Clarity Thesis proceeds in two steps. It first makes a claim about the way in which constitutional structures of government that permit the delegation of policymaking power affect the costs of, and thus voters' access to, information about officials' performance. When institutional arrangements permit delegation, as in the case of separation of powers regimes, and legislators take advantage of that permission, the informational burden on voters increases. Second, the Clarity Thesis makes a claim about how voters perform in low information environments. Where voters are faced with a decision under incomplete information, the Clarity Thesis predicts that they will tend to under-blame legislatures and under-blame agencies. This tendency results in an under-production of electoral accountability.\footnote{Proponents of the Clarity Thesis frequently argue that this tendency to blame legislatures results from legislators exploiting the informational burden voters face by strategically deploying agencies to carry out unpopular policies. According to these blame-shifting accounts, legislators strategically delegate policies that they anticipate to be unpopular to agencies in order to, on the one hand, obfuscate their role in the policymaking process and, on the other, increase the salience of the role of the agency in the policymaking process to the public. But by the logic of the Clarity Thesis, legislators need not engage in strategic behavior to sidestep accountability: the institutional arrangement itself raises the costs of acquiring the relevant information, thus obscuring the details of the way in which policymaking authority is shared and limiting the ability of voters to accurately assign responsibility for policy outcomes. ``Very few voters, regardless of their sophistication, would be able to make much headway in assigning blame or credit in a system in which a dozen or more government branches participated in proposing or vetoing policy initiatives.'' \citep[p. 653]{Nzelibe2010}} This section explains these two steps in more detail.

\subsection{Delegation's Indirect Impact on Blame}

The Clarity Thesis starts with the claim that delegation raises the costs of obtaining information, thus weakening or breaking what Douglas Arnold refers to as the ``traceability chain'' linking officials to policy outputs. \citep[p. 100]{Arnold1990} If voters bearing the costs of a policy seek to hold legislators liable, they must perceive the costs, those costs must be associated with a particular law or policy, and the actors behind the policy must be visible. \citep{Arnold1990} ``Weakening the treceability chain is a superb method for protecting legislators from their constituents wrath for imposing costs on them,'' Arnold argues. \citep[p. 100]{Arnold1990} Constituent wrath, in other words, is lessened when three types of information are unobservable: a policy's political history, its enactment, and its effects. The observability of these three types of information determines a policy's traceability chain. As Nzelibe and Stephenson propose, the information necessary for voters to adequately evaluate politicians is that information concerning ``not only the policy choice and the policy outcome, but also the way in which the policy was adopted: which institutions supported it, which opposed it, whether the President acted unilaterally or with congressional support, [and] whether the proposal died due to lack of presidential support or due to congressional opposition.'' \citep[p. 652]{Nzelibe2010} A policy's traceability chain, then, is composed of three distinct stages in the lifecycle of a policy: (1) the \emph{political history} of a policy's adoption, (2) the \emph{official enactment} of the law or policy, and (3) the \emph{effects} of the law or policy. Delegation, the Clarity Thesis holds, tends to obscure the first two stages. 

These first two stages are grounded in the quite plausible assumption that voters might, in addition to caring about the effects of policy, be concerned with the way in which a policy was adopted: the impetus that lead officials to adopt it, which institutions supported and opposed it, the reasons why officials supported or opposed the policy, and so forth. Delegation, the claim goes, makes it more costly to obtain this information. It does so for two reasons. First, delegation adds complexity to the process of enacting laws and formulating policy. Second, delegation exacerbates the informational asymmetries inherent in representative government. Elected officials, it is thought, acquire information in the course of their employment that places them at an informational advantage vis-a-vis most voters. When policymaking authority is delegated, voters must work at an even greater information disadvantage in assessing the performance of officials.\citep{Gersen2010b} In short, delegation makes it more difficult to figure out which actor supported or opposed any given policy, and their reasons for doing so.

It is worth pointing out how the Clarity Thesis differs from other claims about voter sophistication. Political scientists have long observed that voters tend to be severely uninformed about many issues. As Elmendorf and Schleicher write, ``If there is any well-accepted fact in political science, it is that most voters pay little attention to politics and know little about the basic institutions of government.''\citep[p. 1850-51]{Elmendorf2012} What additional value, then, does the Clarity Thesis add by pointing out one more source of voter confusion? While ``[d]ecades of research show that citizens are often ignorant about politics,'' more recent work has sought to explain voting behavior as a function of voters' responses to certain ``short cuts'' or ``cues.''\cite[p. 3]{James2010} Short cut based voting does not require intimate knowledge of politics, but, rather, a basic sense of which party was in charge and how their own or society's welfare improved or diminished.\citep{Lenz2012b} As Elmendorf and Schleicher explain, ``[s]o long as these voters discern which party is in charge and which is the principal opposition, they can cast a retrospective vote for the governing party or coordinate on an alternative, depending on their sense of local conditions.''\citep[p. 1853]{Elmendorf2012} The Clarity Thesis challenges the first of these two assumptions. Delegation, the Clarity Thesis argues, decreases voters' ability to assess the source of policy. Such an inability tends to undermine the ordinary ways --- short cuts and cues --- in which voters overcome information asymmetries.

Delegation has the effect of obscuring the source of policy because it multiplies the number of individuals and institutions involved in the creation of law and policy. \citep{Gailmard2012} Piecing together the history of a policy requires voters to follow the trajectory of a policy across institutions. At minimum, delegation requires voters interested in the history of a policy's adoption to track two institutions: the legislature and the agency to which power was delegated \citep{Gersen2010b}. These two institutions might be responsive to different sets of constituents, have different policy preferences, or understand the relevant facts differently. Such differences might affect the performance of the institutions. Voters faced with assigning responsibility for an unpopular measure must decide whether the legislature drafted an ill-conceived statute or whether the agency poorly implemented a well-designed idea. If an agency poorly implemented a statute, was that implementation scheme due to the agency's misunderstanding of the facts, or was it simply following the lead of the legislature? What's more, a policy's trajectory may not always proceed so linearly. Policies can be adopted in piecemeal fashion, through a series of negotiations between institutions.

Following a policy's trajectory is particularly difficult because officials typically possess better (although certainly not perfect) information about the state of the world and the consequences of various policy choices than do voters \citep{Shipan2002}. Democratic government is premised on the fragmentation of authority, which produces fragmentation of knowledge. Even absent delegation, authority is partitioned between at least two entities: the people and those they elect. Each partition of authority --- each link in the chain of delegation --- creates an information asymmetry.\footnote{Members of the Senate Finance Committee, for instance, possesses private information about particular cases under investigation, such as facts about the way in which banks or hedge funds supervise employees and manage risk. Second, officials possess expert information about how their chosen actions are likely to affect the world. Even assuming all the facts that the Finance Committee knows about the cases it is considering prosecuting were public, it has a better grasp on how its choices will play out than all but a very small handful of experts outside of government. Finally, officials have private information about their own policy predispositions. Members of the Senate are aware of the motives behind their choice to, say, enact Dodd-Frank. One of the primary tasks of voters is assessing performance in the face of this sort of informational disadvantage. One might think of this as the baseline level of informational asymmetry faced by all democratic regimes, whether or not policymaking power is delegated.} Delegation levers up the asymmetry between the public and officials. It creates new sets of actors with private information, and thus presents voters with second-order informational dilemmas. If an agency poorly implements a policy, for instance, voters must figure out whether the legislature believed that the agency would implement it poorly. If a legislature sends a well-designed regulatory scheme to an agency to implement, and the agency skimps on the resources necessary to implement it, should the voter infer that legislature knew that the agency would fail to fund the project? Delegation, then, exacerbates the information asymmetries inherent in representative government.\footnote{Consider the case of the Securities and Exchange Commission. Agency officials possess better information about the world than the public and, according to some accounts, better information than the legislature vis-a-vis financial affairs. The SEC, for instance, possesses private information about the civil cases it considers pursuing and the effect of those cases on the capital markets. Even if all the facts that the SEC knows about the cases it is considering prosecuting were public, the SEC, through institutional knowledge, has a better grasp on how its choices will play out in the course of legal proceedings than all but a very small handful of experts outside of government. Finally, agency officials know their own motives and intentions better than does the public and legislature.} 

The second type of information important for voters' ability to assign responsibility is information about the policy choice itself. Unlike information about a policy's history, this type of information is public. In the case of legislatures, it is available in the text of the statute and, in the case of agencies, it is available in the Federal Register or through less formal publications, such as the agency's website. Despite the relative ease of accessing this information, the clarity thesis observes that delegation does, in fact, increase the quantity of the information that the voter must process in order to accurately attribute responsibility. In particular, voters must determine what the legislature enacted itself, and what it left for the agency to carry out. If this information becomes costly to obtain, voter performance suffers. 

The Clarity Thesis is particularly concerned with delegation's effect on information concerning a policy's enactment or political history. Thus, the first round of experiments varies the observability of these two types of information, and always makes the effects of a policy observable to participants. The Clarity Thesis predicts that participants who cannot observe this information will tend to blame the legislature.

\subsection{Delegation's Direct Impact on Blame}

The Clarity Thesis maintains that \emph{if} voters possess these three types of information necessary to trace a chain of political responsibility --- (1) the \emph{political history} of a policy's adoption, (2) the \emph{official enactment} of the law or policy, and (3) the \emph{effects} of the law or policy --- then delegation does not impact voter performance. Recent work in experimental economics, however, suggests that delegation affects attributions of blame even in cases where subjects possess full information about the history of an outcome. This literature suggests that the Clarity Thesis explains only part of relationship between delegation and blame. The remainder of the explanation, this literature suggests, stems from the effect \emph{indirect agency}. The second round of experiments reported here tests whether delegation has type of direct impact on blame suggested by this prior experimental literature. If it does, then the Clarity Thesis will need to be modified accordingly.

The experimental economics studies reveal an asymmetry between individuals' perceptions of direct and indirect action. \citep{Coffman2011} Norm violators face fewer attributions of blame, and less punishment, when they intermediate their relationship with the victim of the norm violation. \citep{Paharia2009, Bartling2012} Principals that carry out an act via an agent tend to face less blame in the context of economic games, like the dictator and ultimate games,  \citep{Bartling2012} and in the context of market-driven business transactions.  \citep{Paharia2009} This result holds even where the agent's interests are aligned with the principal's interests. \citep{Gneezy2001} Using vignettes, Paharia et al. show that firms that outsource to a business partner face less blame than do firms that produce the same outcome directly. \citep{Paharia2009} In the context of market mechanisms, attributions of blame appear to be sensitive the process or structure that produces an outcome. \citep{Hayashi2013, Wiltermuth2011} Notably, these results do not depend on hiding or otherwise obscuring information from subjects. In the dictator and ultimatum games testing the affects of delegation on blame, subjects observe the incentive structure facing each participant and the actions that they take. Likewise, in the vignette-based experiments, subjects are informed of the details of the principal-agent relationship and the incentives facing each actor.\footnote{Of course, it is impossible to provide subjects with theoretically perfect information. Experiments that seek to provide complete information can, at best, provide the range of information likely to affect subjects' judgments.} In short, this literature suggests that, even where individuals possess all three types of information that the Clarity Thesis identifies as necessary and sufficient to correctly allocate blame, delegation may nonetheless induce voters to under-blame legislatures and over-blame agencies.

The second round of experiments tests whether delegation directly impacts blame. 
The experiments reported here present participants with precisely the three types of information that the Clarity Thesis predicts is necessary and sufficient to produce accurate assessments of blame. In other words, participants read exactly the sort of information, the absence of which, is thought to lead to inaccurate assignments of responsibility. If, after presenting participants with this information, we nonetheless find inaccurate assignments of responsibility, one cannot reject the Clarity Thesis' argument as to the indirect effects of delegation. Indeed, the experiments reported in Round 1 support this argument. However, such results would tend to support the claim that delegation might trigger some cognitive bias, which tends to result in participants under-blaming the legislature and over-blaming the agency. 

\section{Round 1: Does Delegation Indirectly Impact Legislative Blame?}

The first round of experiments reported here tests the claim that delegation \emph{indirectly} induces voters to over-blame legislatures. The Clarity Thesis holds that delegation indirectly affects voter behavior by raising the costs of information, and thus increases the uncertainty as to the relative responsibility of the legislature and agency. Voters, in response to this low information environment, tend to resolve the ambiguity in favor of the legislature. Experiments 1a and 1b test whether providing respondents with information concerning (1) the role of the legislature and agency in the policy's political history, (2) the institution responsible for the policy's formal enactment, and (3) the effects of the policy on the electorate impacts blame attributions. Does varying each of these three types of information change the quantity of blame participants allocate to the legislature?  

Experiments 1a and 1b employ vignettes that tell the story of a jurisdiction's response to a serious problem --- a natural disaster, for instance --- in concise (350 words or less), non-technical language. In order to measure the effect of information on the quantity of blame participants allocate to the legislature, each participant was randomly assigned to one of four conditions. In order to establish a baseline level of blame, one group of participants were assigned to a direct condition. This group read a vignette describing a wholly legislative response, which did not involve an agency, to the problem. The remaining participants were assigned to one of three indirect conditions. In the indirect conditions, the legislature utilizes an agency to carry out the response. The three indirect conditions, however, contain varying quantities of information about the agency's role in the response. In the first indirect condition, participants read a vignette that contains only information about the \emph{effect} of the response. The respective roles of the legislature and agency are entirely opaque to the participant. In the second indirect condition, participants read a vignette that contains information about the effect of the response and history of the \emph{official enactment}. In this vignette, the participant is aware not only of the effects, but also is aware of which institution --- the legislature or the agency --- made the decision to enact the policy. Finally, in the third indirect condition, participants read a vignette that contains information about the effect of the response, the official enactment, and the policy's \emph{political history}. Only this third indirect condition contains the full quantity of information the Clarity Thesis argues is necessary and sufficient to accurately assess blame. 

\subsection{Experiment 1a}

Experiment 1a was administered online to 248 participants, recruited and compensated via Amazon's Mechanical Turk web services. All participants completed two demographic questions (41 percent female, $M_{age}$=33) and were randomly assigned to one of four conditions described above: direct legislative action, $indirect_{effects}$ legislative action, $indirect_{eff+enact}$ legislative action, $indirect_{all}$ legislative action. 

In order to test the Clarity Thesis' claim that delegation indirectly affects blame, Experiment 1a compares the blame scores in each of the three indirect conditions with the baseline blame score established by the direct condition. In the indirect conditions, the legislature employs a “mere tool” or “instrument” agency.\citep[p. 153]{Kutz2000} It thus seeks to eliminate any effects resulting from factors such as perceived lack of foreknowledge of the agency’s intentions or lack of control over the agency.\citep{Paharia2009} In order to further emphasize the fact that the agency is a mere tool of the legislature, the vignettes in Experiments 1a and 1b are structured such that the legislature's makes a choice that renders the agency able to act in precisely one way. More concretely, in both experiments' indirect conditions, the legislature promises to convey funds to address some social problem, but then reneges on that promise and channels the funds elsewhere. The agency, whose job it was to distribute those funds, is thus left with no option but to fail to address the social problem, since it has no funds with which to work. The resulting vignette is one in which the actions available to the mere instrument agency are completely determined by the legislature. The agency must take the unpopular action. 

As a result of the legislature's broken promise, ``large parts of the city remain devastated'' and ``[s]ignificant numbers of residents still lack basic services like electricity and hot water.'' Participants were then asked, ``On a scale of 1 (not blameworthy at all) --- 10 (very blameworthy), how blameworthy do you think the legislature's action was?''\footnote{In all experiments reported here, subjects entered their judgment using a horizontal sliding scale, with ticks at the points from 0 to 10. \emph{See} www.qualtrics.com/university/researchsuite/basic-building/editing-questions/question-types-guide.} The Clarity Thesis predicts that, as participants' information concerning the respective roles of the legislature and agency increases, legislative blame scores will rise. Since participants in the direct condition and the $indirect_{all}$ condition possess complete information, we should see equivalent levels of blame in these two conditions. 

Participants assigned to the $indirect_{effects}$ condition read a vignette in which the legislature delegates responsibility for the disaster response to an agency.

\begin{quotation}

The legislature then delegated responsibility for managing the clean-up efforts to SQL, an official government agency with expertise in handling emergencies. The legislature felt comfortable delegating responsibility for this important task to SQL because it exerts high levels of control and oversight over the agency. SQL's primary job was distributing the money that the legislature promised to the city. 
 
SQL, however, never distributed the money. As a result, large parts of the city remain devastated. Significant numbers of residents still lack basic services like electricity and hot water.

\end{quotation}

Participants assigned to this condition have access to information about the institutions involved in the policy, and the effects of that policy, but no information concerning how or why the policy was established. In other words, the $indirect_{effects}$ condition omits information concerning the two types of information the Clarity Thesis deems crucial to accurately allocating blame. 

By comparison, participants assigned to the $indirect_{eff+enact}$ condition read a vignette identical to the $indirect_{effects}$ condition, except that it contained information concerning which institution --- the legislature or the agency --- had, in fact, enacted the measure. In the $indirect_{eff+enact}$ condition, participants read that ``[a]fter consulting with SQL, however, the legislature enacted a bill re-directing the funds.'' Thus, in the $indirect_{eff+enact}$ condition, possess one of the two types of information the Clarity Thesis deems crucial to accurately allocating blame. 

Finally, in the $indirect_{all}$ condition, participants read a vignette that contained all of the information in the $indirect_{eff+enact}$ condition, plus information concerning the political history of the policy. The political history is identical to the history set out in the direct condition: the legislature believed it was in its electoral interests to distribute the clean-up funds elsewhere, so it abandoned the clean up efforts. Recall that the legislature ``exerts high levels of control and oversight over the agency,'' and, since the legislature made the decision to re-direct the funds away from the agency, the agency could not have conveyed the funds to the city. The effect, then, is that in the $indirect_{all}$ condition the agency is a mere instrument of the legislature. 

As shown in Figure 1, as the quantity of information increases, so does the accuracy of the blame allocated to the legislature, as compared with the baseline blame score established in the direct condition (${M}_{Direct}$ = 8.929, \emph{sd} = 1.455). Participants who read the $indirect_{effects}$ condition (${M}_{Direct}$ = 5.86, \emph{sd} = 2.75) rated the legislature as significantly less blameworthy than participants who read the $indirect_{eff+enact}$ condition (${M}_{Direct}$ = 7.72, \emph{sd} = 2.24). Adding the third type of information in the $indirect_{all}$ condition (${M}_{Direct}$ = 8.26, \emph{sd} = 2.51) further raises blame scores. Analyzing Experiment 1a using a Welch two-sample \emph{t}-test reveals significant differences between the direct condition and the $indirect_{effects}$ condition, as well as between the $indirect_{effects}$ condition and the $indirect_{eff+enact}$ condition. 

Thus, consistent with the Clarity Thesis, these results support the claim that participants' accuracy improves as more information is revealed. Low information environments tend to advantage legislatures in cases where the legislature deployed a mere instrument agency to deflect blame. 


%\begin{figure}[htb]
%\begin{center}
%\includegraphics[height=3in,width=4in]{Exp2plot.png}
%\end{center}
%\end{figure}


\subsection{Experiment 1b} 

Experiment 1b employs the same structure as Experiment 1a. The legislature promises to commit funds to a social problem, but then reneges on that reason for selfish reasons. The agency is thus left with no option but to fail to address the social problem, since it has no funds with which to work. Experiment 1b was administered online to 212 individuals, recruited and compensated through Amazon's Mechanical Turk. 

After completing two demographic questions (45 percent female, ${M}_{age}$=32), all subjects read a scenario describing how ``Last year on election day, a malicious computer virus sabotaged the voting machines in jurisdiction B.  As a result, the legislature promised to overhaul the voting machines by next year's elections.'' As in Experiment 1a, participants assigned to the direct condition read a scenario describing a wholly legislative response.

\begin{quote}

After considering the matter more carefully, however, the legislature realized that overhauling the voting machines could open the door to overhauling other aspects of the electoral process. This prospect worried the incumbent legislators, since they knew how to win in the current electoral environment. Thus, rather than overhaul the voting machines, the legislature decided to use the money for other purposes, including studies showing how well other aspects of the electoral system worked. The legislature enacted a bill diverting the voting machine money to these other purposes. 

As a result, the voting machines remain susceptible to computer viruses. The compromised integrity of the voting machines has induced many to lose faith in the quality of the elections in jurisdiction B. 
 
\end{quote}

The remaining participants were randomly assigned to one of three indirect conditions, corresponding to the three indirect conditions in Experiment 1a. In the $indirect_{effects}$ condition, participants read a vignette that stated that the legislature ``delegated responsibility for managing the overhaul to PPW, an official government agency with expertise in elections.'' Because the $indirect_{effects}$ condition hides information concerning the policy's enactment and political history, participants in this condition know that ``PPW, however, never purchased or installed the new machines,'' but do not know why this is the case. In the $indirect_{eff+enact}$ condition, participants read a vignette that explains that  ``the legislature enacted a bill diverting the funds.'' Finally, in the $indirect_{all}$ condition, participants read a vignette that explains the political history leading to the legislature's decision, as set out in the baseline condition. 


\begin{figure}[htb]
\begin{center}
\includegraphics[height=3in,width=4in]{Exp1aplot.png}
\end{center}
\end{figure}

As in Experiment 1a, participants who read the direct scenario (8.96, \emph{sd} = 1.73) found the legislature significantly more blameworthy than the $indirect_{effects}$ condition (6.4, \emph{sd} = 2.54). Blame continued to rise as participants read vignettes with more information ($indirect_{eff+enact}$ = 7.74 \emph{sd} = 2.24; $indirect_{all}$ = 8.6, \emph{sd} = 2.13).

\subsection{Round 1 Summary}

The experiments reported here suggests that delegation indirectly affects blame, as predicted by the Clarity Thesis. The experiments in Round 1 test whether participants evaluating the indirect actions of a legislature more accurately assess blame when they possess more information concerning a policy's enactment and political history. As predicted by the Clarity Thesis, participants tended to resolve informational ambiguities in favor of the legislature. Delegation, then, would appear to benefit the legislature in two respects. First, delegation tends to benefits legislatures to the extent that they are primarily responsible for unpopular policy. Even where legislatures do not seek to shift the blame to the agency, voters will tend to treat unobservable information showing the legislature should shoulder responsibility as information implicating the agency. One need not attribute any failure of good faith by voters in such a situation; rather, this tendency to resolve informational ambiguities in favor of the legislature might result from a tendency by voters (i.e., the political principals in a democracy) to favor an agents that they selected, rather than agents that they did not select. Second, where legislatures \emph{do} seek to shift the blame for an unpopular policy, they can accomplish this feat, at least in part, simply by \emph{introducing} an agent into the process. If delegating to an agent tends to lower the visibility of the respective causal roles, a legislature seeking to shift the blame need not publicly exaggerate the agency's role; voters will do that.  

\section{Round 2: Does Delegation Directly Impact Legislative Blame?}

In Round 1, participants performed best when they possessed information concerning the respective roles of the legislature and agency. The stronger the traceability chain, the better the performance. Even with advantage of the strongest traceability chain, however, (i.e., in the $indirect_{all}$ condition) participants on average rated the legislature acting through a mere instrument agency as less blameworthy than a legislature acting directly. This data fits uncomfortably with the Clarity Thesis. While the results in Round 1 support the Clarity Thesis' claim that voters tend to resolve informational ambiguities in favor of the legislature, those results do not support the claim that full information results in accurate assessments of blame. This position aligns with prior work that show that blame evaluations are a function of the degree of control and foresight an individual possesses. \citep{Cushman2012} Nonetheless, even with full information, participants on average rated the legislature acting through a mere instrument agency as less blameworthy than a legislature acting directly. Even where the legislature possessed a high degree of control over the agency, and foresaw that the agency would comply with its preferred policy position, participants nonetheless assigned it less blame than they assigned to a legislature enacting the same policy directly. Does delegation \emph{directly} impact blame? 

The second round of experiments tests whether, consistent with the results on delegation and blame in the experimental economics and moral psychology literatures, delegation reduces attributions of blame even in those cases where subjects possess information about the way in which a policy was adopted. In order to do so, it employs vignettes, which set out, in concise (350 words or less), plain language, the history of a policy jointly produced by a legislature and administrative agency. Each vignette tells the story of some problem the voters of a jurisdiction faced --- increased pharmaceutical costs, for instance --- and the way in which the legislature and agency addressed that problem. In so doing, the narrative explains ``the way in which the policy was adopted'' and the policy choices made by the legislature and agency \citep{Knutson2010}. Because the vignettes are clear, accessible, and present the precisely the sort of facts that the Clarity Thesis argues are determinative of voter behavior, these experiments provide clean tests of the Clarity Thesis' claim that delegation primarily affects blame by creating informational burdens.\footnote{They provide one, although certainly not the only, test of whether any apparent irrationality or misappropriation of blame is necessarily explained by informational burdens. A full accounting off the question would present a variety of fact patterns testing a variety of dimensions thought likely to affect voter behavior.}

Experiments 2a and 2b each modify the structure of the experiments in Round 1 in order to further test whether delegation directly impacts blame. Experiment 2a seeks to nudge participants toward more ``reason-based choice'' by presenting them with both direct and indirect scenarios. \citep{Hsee2010} Whereas Experiments 1a and 1b employ ``separate evaluation'' --- a between-subjects design in which participants assessing the blameworthiness of indirect legislative action do not \emph{also} assess the blameworthiness of direct legislative action --- Experiment 2a employs ``direct evaluation,'' in which participants read both indirect and direct vignettes. By presenting participants with both types of vignettes, Experiment 2a might nudge participants toward treating the direct and indirect conditions more similarly. Experiment 2b addresses a potential objection; namely, that even the $indirect_{all}$ conditions still lack information relevant to tracing a chain of responsibility for the policy. Thus, Experiment 2b provides even \emph{more} information to participants than is contained in the $indirect_{all}$ conditions in Experiments 1A and 1B. 

If, after modifying the structure of the experiments in Round 2, participants treat direct and indirect legislative action as equivalently blameworthy, then one might explain the difference between the blame scores in Round 1's $indirect_{all}$ conditions and Round 1's direct conditions as an artifact of the experimental design. If, however, voters find that the indirect action type is less blameworthy even with complete information as to the history of the policy's adoption, then it would support the claim that delegation, in addition to creating informational burdens, \emph{directly} affects voter behavior. 

\subsection{Experiment 2a}

Experiment 2a employs a version of what Hsee and Zhang refer to as ``joint evaluation'' \citep{Hsee2010}. In experiment employing joint evaluation, ``multiple alternatives are simultaneously available for consideration [by participants], while under separate evaluation only one alternative is considered.'' \citep[p. 2]{Paharia2009} The finding underlying this design is that joint evaluation generally ``allows for the consideration of attributes that are difficult to assess in isolation and is therefore more likely to result in reason-based choice ... .'' \citep[p. 2]{Paharia2009} Thus, Experiment 2a employs a within-subjects design, where all participants first read the direct condition, then read the indirect condition. In the direct condition, participants were asked to attribute blame to the legislature. In the indirect condition, participants were asked to assess the blameworthiness of both the legislature and the agency. This design might elicit a more reasoned comparative assessments of the legislature versus the mere instrument agent than the separate evaluation featured in Round 1.\footnote{In the Paharia et al. study, the joint evaluation occurs through asking participants which condition, the direct or indirect one, is more blameworthy. Experiment 2a does not operationalize joint evaluation in this manner because of external validity concerns. Voters do not evaluate two versions of the same event, one direct and one indirect, but, rather, evaluate the event that occurred.}

Experiment 2a was administered online to 142 individuals, recruited and compensated through Amazon's Mechanical Turk. After completing two demographic questions (47 percent female, ${M}_{age}$=36.9), all subjects read a scenario describing how ``hackers unleashed a malicious virus on the computerized trading systems used by large companies in several jurisdictions,'' causing great harm to their respective financial systems. ``In response, the legislature of each jurisdiction passed a bill requiring computerized trading systems to comply with a new set of security measures,'' which were ``very expensive to implement.'' To encourage rapid adoption, legislatures promised to reimburse companies that implemented the new system within six months. Each jurisdiction, however, ``chose different ways of enforcing the law.'' All subjects were presented with accounts of how the enforcement process proceeded in two of these jurisdictions. In the direct condition (``State A''), the legislature reneged on its promise to reimburse the companies, and, as a result, ``many companies went out of business, and those that had not yet implemented the costly measure decided not to do so.'' All subjects were then asked ``On a scale of 1 (not blameworthy at all) --- 10 (very blameworthy), how blameworthy do you think the legislature's action was?'' In the indirect condition, the legislature ``delegated the task of enforcing the law to SECRA, which is an official administrative agency, created by the legislature to serve as its agent.'' Subjects were then told that the legislature ``exercises a relatively high degree of control over'' the agency. In the indirect condition, ``the legislature did not make the funds available to SECRA,'' leaving the agency ``unable to reimburse the companies for the upgrades they made.'' As a result, the same consequences followed. All subjects were then asked, ``On a scale of 1 (not blameworthy at all) --- 10 (very blameworthy), how blameworthy do you think SECRA's action was? How blameworthy was the legislature's delegation to SECRA?''

Experiment 2 was analyzed using a Welch two sample paired t-test. Participants who rated the direct scenario (${M}_{Direct}$ = 8.51, \emph{sd} = 1.81) significantly more blameworthy than the indirect scenario (${M}_{Indirect}$ = 6.70), \emph{sd} = 3.37). The results are significant (\emph{t} = 6.45, \emph{p} = $<$.001). Thus, even where the agency's hands are tied by the legislature's policy decisions, and where participants are able to jointly evaluate the direct and indirect acts, delegation appears to directly impact attributions of blame. 

\begin{figure}[htb]
\begin{center}
\includegraphics[height=3in,width=4in]{Exp2aJELS.png}
\end{center}
\end{figure}

Together, Experiments 1a, 1b, and 2a challenge the Clarity Thesis' claim that information is the exclusive way in which delegation impacts voter behavior.\footnote{The results in Experiment 2a also challenge the proposition, found in the Paharia, et al. piece, that joint evaluation produces rational evaluations.} While delegation might obscure information voters would find relevant, policies that result from delegated arrangements appear to diminish the blame legislatures face even where voters are presented the history of a policy's adoption and the relevant choices made by the legislature and agency. While counterintuitive, these results accord with prior studies that reveal delegation to diminish blame in the context of economic games and business transactions.  

\subsection{Experiment 2b}

One might object to the claim that delegation directly influences blame by arguing that the vignettes in Experiments 1a, 1b, and 2a omit material information about the degree of control and foresight that the legislature possesses over the agency. Such an omission might induce participants to resolve informational ambiguities in favor of the legislature. As a check against this concern, Experiment 2b layers more information on top of the information contained in the $indirect_{all}$ hurricane disaster vignette employed in Experiment 1a. Experiment 2b's $indirect_{plus}$ condition adds information concerning the nature of the legislature's power over the agency, as well as information concerning the agency's role in the unpopular policy decision. Experiment 2b then compares participants' blame scores in the $indirect_{all}$ condition with the $indirect_{plus}$ condition.

The $indirect_{plus}$ vignette reads in relevant part as follows, with the added information italicized. 

\begin{quote}

Last month, a terrible hurricane ravaged a large city on the western coast of jurisdiction B. In the immediate aftermath of the storm, the legislature promised the city to pay for all of its clean-up costs.
 
The legislature then delegated responsibility for managing the clean-up efforts to SQL, an official government agency with expertise in handling emergencies. SQL's primary job was distributing the money that the legislature promised to the city.  The legislature felt comfortable delegating responsibility for this important task to SQL because it exerts high levels of control and oversight over the agency. \emph{For instance, the legislature can remove the head of SQL at will. Further, unwritten conventions empowered the legislature to exert control over the agency's use of discretion.} 

After considering the matter more carefully, however, the legislature realized that the residents of the city were unlikely to ever make substantial donations to their future re-election campaigns. "Wouldn't it be better," the legislators reasoned, "to direct the funds meant for the disaster relief to other voters more likely to help us in the future?" \emph{SQL told the legislature that doing so would significantly hamper the city's recovery, but it was ineffective in persuading the legislature.} 
 
Thus, the legislature enacted a bill re-directing the funds. As a result, large parts of the city remain devastated. Significant numbers of residents still lack basic services like electricity and hot water.

\end{quote}

Participants receiving the $indirect_{plus}$ condition read a vignette that contains all of the information contained in the $indirect_{all}$ vignette, \emph{plus} more information concerning the legislature's control over the agency and the agency's role in formulating the unpopular policy. The added information further demonstrates the agency's lack of control, and portrays the agency as opposing the unpopular policy.

Despite the added information, participants reading the $indirect_{plus}$ vignette do not find the legislature more blameworthy. In fact, participants reading the $indirect_{plus}$ vignette found on the legislature on average slightly less blameworthy ($indirect_{plus}$ = 8.6, $indirect_{all}$ = 8.20). The difference, however, is not statistically significant (\emph{t} = 1.25, \emph{p} = 0.21). Experiment 2b, then, suggests that adding more information does not alter participants' attributions of blame.\footnote{In fact, adding more information than is contained in the $indirect_{all}$ condition may confuse participants, which would explain the lower blame scores in the $indirect_{plus}$ condition.}

In sum, Experiments 1a, 1b, and 2a suggest that policies that result from delegated arrangements appear to diminish the blame legislatures face even where voters are presented the history of a policy's adoption and the relevant choices made by the legislature and agency. While counterintuitive, these results accord with prior studies that ``revealed a moral preference for indirect agency'' under certain conditions.\citep[p. 7]{Paharia2009} Experiment 2b addresses one possible objection to these results, namely, that the vignettes lack crucial information concerning the extent of legislative control and foresight. In response to this objection, Experiment 2b layers more information concerning legislative control and foresight on the hurricane response vignette. Blame scores, however, do not rise when this new information is included. Consistent with prior research on cue-taking and information overload \citep{Lupia1994b}, this result suggests that the quantity of information contained in the $indirect_{all}$ condition is sufficient for participants to trace a chain of responsibility for the policy.\citep{Arnold1990}

\section{Round 3: Does Delegation Impact Agency Blame?}

The first two rounds of experiments suggest that delegation both directly and indirectly impacts attributions of \emph{legislative} blame. Proponents of the Clarity Thesis sometimes argue that this tendency to soften legislative blame is an unalloyed good for legislators. Legislatures might well care about how much collateral damage agencies will face, however. Even if legislatures are interested in the fates of agencies only in order to take the blame for unpopular policies, they still need legislatures to possess some popular legitimacy. Less cynically, legislatures rely on agencies to accomplish things, even if they need them to take the fall in this instance. Agencies are, over the long run, valuable partners. If agencies are to manage blame, they need to have an idea of how blame avoidance will impact their partners' fate. Thus, we need to know how a blame-shifting strategy affects agencies.

The results considered thus far suggest that the Clarity Thesis is correct in arguing that delegation reduces attributions of legislative blame, but incorrect in suggesting that the exclusive or primary causal mechanism underlying the effect is informational. Rather, legislative blame declines even where information is complete. But what of \emph{agency} blame? Fiorina argues that delegation allows legislators to exploit informational asymmetries in order to ``disguise their responsibility for the consequences of the decisions ultimately made'' and shift blame to agencies. \citep[p. 47]{Fiorina1982} More recently, Justin Fox and Stuart Jordan have set out the necessary preconditions for blame shifting, one of which is that ``politicians must have more information than voters about the actions the bureaucracy is likely to take with any authority delegated.'' \citep[p. 843]{Fox2011} When these preconditions exist, legislators can strategically delegate to agencies in order to shift blame for unpopular policies. In sum, misattributions of agency blame, according to the Clarity Thesis, tend to result in \emph{over}-blaming agencies. Agency blame, on this view, is a one-way ratchet. 

The third round of experiments tests whether misattributions of agency blame might occur absent incomplete information, and whether the misattributions always involve over-blaming agencies. Further, it tests these claims under a variety of institutional arrangements. The Clarity Thesis typically fails to specify two crucial features of the legislature-agency relationship: (1) How much oversight or control the legislature possesses over the agency, and (2) how much policymaking authority the agency possesses. Round 3 varies these conditions in order to gauge their influence on blame. Doing so allows us to address questions such as whether voters are more punitive when legislatures delegate more power to agencies. More generally, introducing these two conditions helps us to draw a larger picture of the way in which diverse institutional arrangements affect attributions of blame. 

\subsection{Legislative Control over Agencies}

Political scientists have long debated the question of how legislatures best constrain agencies \citep{Lupia1994}. One of, if not the primary, ways in which legislatures exert control over agencies is through oversight \citep{Barkow2010}. Schwartz and McCubbins distinguish, classically, between two modes of agency oversight: police patrols and fire alarms. Legislatures engage in police patrol oversight when they actively monitor agency activity. Active monitoring, however, is costly, and so they may only get involved when the agency draws the ire of some third party constituency. Only then does the legislature seek to put out the agency's fire. 

Experiment 3 incorporates a mix of police patrol and fire alarm style oversight. Unlike the experiments in Rounds 1 and 2, which tested only instrument agencies, Round 3 includes autonomous agencies. The vignettes state whether the legislature possesses a high or low degree of oversight over the agency. It then illustrates this oversight with an example. Thus, autonomous agencies are defined by two features in this experiment, corresponding to the two features that define instrument agencies. Legislatures exercise comparatively less control and oversight over autonomous agencies, and, in order to illustrate this autonomy to participants, autonomous agencies act against the preferences of the legislature that delegated power to them. Instrument agencies, by contrast, are monitored and comply with legislative preferences. 

\subsection{Policymaking Authority}

The Clarity Thesis addresses the role of the breadth of the agency's authority only insofar as it is argued that \emph{broad} delegations of authority provide the most insulation for legislators. Fiorina, for instance, argues that legislators can provide agencies with broad grants of authority, without explicit, \emph{Schecter Poultry}-style standards, in order to shift responsibility for unpopular decisions to agencies. Schoenbroad argues that Congressional legislation that features ``attractive abstractions,'' such as instructions to agencies that they maintain ``orderly markets,'' help legislators avoid blame. Fox and Jordan argue that legislators can avoid blame by turning over policy decisions wholesale to agencies that they believe share their preferences.\footnote{On Fox and Jordan's account, legislators avoid blame by exploiting an informational asymmetry: they know the political preferences of agencies, but the public takes agencies to be neutral experts acting in the public interest. Thus, to delegate full policymaking authority is to delegate more of the decision to an entity committed to acting on the public's behalf.} 

Experiment 3 tests the idea that broad delegation leads to more blame under conditions of full information. All things equal, one might hypothesize that, the more policymaking authority an agency possesses, the more blame voters will be willing to assign them. In order to test this hypothesis, Experiment 3 was structured around three types of vignettes, which correspond to three types of policymaking authority that an agency might possess \citep{Freeman2000}. In the Transmission type \citep{Stewart1975}, the legislature simply delegates the power to enforce or apply norms. Such authority might be seen as ``a relatively lower-order activit[y].'' \citep[p. 30]{Freeman2000} In this sense, enforcement authority exists at the bottom of the policymaking hierarchy. In the Implementation type, the agency possesses the authority to ``translate'' norms or standards into regulated practices \citep{Freeman2000}. Finally, at the top of the policymaking hierarchy is the power to create norms. In the Creation vignette, the agency possesses the legislative-like authority to author norms. In comparing attributions of blame for these three types of policymaking authority, Experiment 3 tests whether agencies with more power to control the content of a policy receive more blame. 

\subsection{Procedure and Results}

The study was administered to 312 subjects (42 percent female, $M_{age}$=31.8), recruited and compensated through Amazon's Mechanical Turk. Participants read three scenarios, corresponding to the three types of policymaking authority just mentioned. In each scenario, a multijurisdictional problem confronts the legislatures of several states. Participants read how the legislature of State A responded to the problem directly (the Baseline scenario), and then read how the legislature of State B responded to the problem indirectly by delegated some authority to an agency. In the direct scenario, the legislature responds to the impetus by enacting a law. In the indirect scenarios, the legislature responds to the impetus by enacting a law and delegating either Transmission, Implementation, or Creation authority to an agency. The acts taken, and the consequences that follow, are the same in the direct and indirect scenarios.

In the Transmission scenario, which is substantially similar to the vignette employed in Experiment 1, the impetus for political action is a breakdown of an important brokerage service, which threatens stock markets. The brokerages can be updated with costly repairs, for which the legislature promises to reimburse the companies. In fact, the (legislature or agency) reneges on its promise, and a financial fallout ensues. In the Implementation scenario, a pro-environment ideological shift results in heightened pollution regulations, which creates a demand for hands-on regulatory implementation. However, the (legislature or agency) fails to meet with all but the most profitable companies, and, as a result, drafted cost-insensitive regulations that ended up not having less of an impact than they might have. In the Creation scenario, an opportunity to improve coordination in the shipping industry could lower pharmaceutical prices, if uniform standards are established. Despite this opportunity, the (legislature or agency) adhered to an ideology of experimentation over uniformity, and prices remained high. 

For each scenario, all participants read the direct condition (legislative action), which established a baseline level of blame. For each of the three indirect scenarios, participants were placed in an Instrument condition or Autonomous condition.\footnote{More specifically, for each of the three scenarios, participants were randomly assigned to one of four conditions: Instrument, Autonomous, Mixed-High, or Mixed-Low. Based on planned comparisons, I analyzed only the Instrument and Autonomous conditions. The Mixed conditions cross the level of control the legislature possesses over the agency and whether the agency violated the legislature's policy preferences. In the Transmission scenario, the mean blame scores are 8.43 (mixed-high) and 4.04 (mixed-low). In the Implementation scenario, the mean blame scores are 8.29 (mixed-high) and 6.03 (mixed-low). Finally, in the Creation scenario, the mean blame scores are 7.48 (mixed-high) and 7.12 (mixed-low).} As in Experiment 2, in the Instrument condition the legislature exerts high levels of control over the agency and, in the course of the vignette, the agency in fact complies with policy preferences of the legislature, which is intended to reinforce the legislature's control. In the Autonomous condition, on the other hand, the legislature is described as possessing relatively little control over the agency and, in the course of the vignette, the agency violates the legislature's policy preferences. 

The degree of legislative control over the agency ends up making a significant difference in the blame participants parceled out. In the three direct scenarios, in which the legislature enacts policy without delegating, participants gauged the legislative blame to be 8.422 in the Transmission scenario, 8.274 in the Implementation scenario, and 8.546 in the Creation scenario. When legislatures delegated the task to an instrument agency, participants blamed the instrument significantly less, as hypothesized. Participants found the acts of autonomous agencies to be much more blameworthy than they did instrument agencies. 


\begin{figure}[htb]
\begin{center}
\includegraphics[height=3in,width=4in]{Exp3plotdiff.png}
\end{center}
\end{figure}

The divergence in the blame faced by instrument agencies and autonomous agencies suggests that delegation does not uniformly affect agency blame. Much as participants found autonomous agencies more blameworthy than instrument agencies, so too did participants find those agencies that were delegated more extensive policymaking authority more blameworthy. As policymaking authority increases, the difference between the baseline scenario and the indirect scenario decreases. In the Transmission scenario, the difference between the baseline case and the indirect conditions (the average of the instrument agency blame score and the autonomous agency score) is 2.335. In the Implementation scenario, the difference drops to 1.885. In the Creation scenario, the difference falls to 1.140. If one defines the error score as the difference between the paired baseline blame score and the agency blame score, then one observes that the error score monotonically decreases. The declining difference between the baseline scenario blame score and the indirect blame score reflect increased attributions of responsibility to those agencies possessing more policymaking authority. This increased attribution of responsibility is, this paper hypothesizes, a result of the increased role the agency possesses in determining the content of the resultant policy. 


\begin{figure}[htb]
\begin{center}
\includegraphics[height=3in,width=4in]{Exp3for3.png}
\end{center}
\end{figure}

In addition to running Experiment 3 on Mechanical Turk, I also ran Experiment 3 on a population of students and staff at the Experimental Social Science Laboratory at UC-Berkeley. 96 subjects participated (63 percent female, $M_{age}$=26.5) These results follow the same pattern as the results obtained through the Mechanical Turk subject pool. The degree of legislative control over the agency significantly impacts the quantity of blame the agencies face. In the three direct scenarios, in which the legislature enacts policy without delegating, participants gauged the legislative blame to be 8.43 in the Transmission scenario, 7.94 in the Implementation scenario, and 7.58 in the Creation scenario. As in the Mechanical Turk subject pool, where legislatures delegated the task to an instrument agency, participants blamed the instrument significantly less (in each of the three cases, \emph{p} = $<$.05). However, participants found autonomous agencies to be much more blameworthy. 

\section{Discussion} 

The present study suggests that evaluations of policy are not independent of the structures that produce them. Rather, delegations of policymaking authority systematically induce voters to under-assign blame to legislatures. Legislatures face less blame even in cases where they delegate to non-autonomous, ``instrument'' agencies. In effect, legislatures can delegate the final, public-facing component of a policymaking process to an agency in order to soften the blame they face. This finding coheres with a body of economic research that shows that attributions of blame diminish when a principal employs an agent \citep{Coffman2011}.

Are we more adept at assigning blame to agents? This question can be analyzed along two dimensions: comparative evaluations and point estimates. In the realm of comparative evaluation, participants accurately assign more blame to those agencies exercising more policymaking authority and accurately assign more blame to those agencies operating under less legislative oversight. Participants accurately assigned more blame to agencies as their policymaking authority increased. Agencies that created unpopular norms were gauged more blameworthy than those that implemented unpopular norms; and agencies that merely enforced or applied unpopular norms were found less blameworthy still. These findings suggest that voters are able to identify degrees of policymaking authority and tailor their blame accordingly. 

While participants accurately discern comparative responsibility, they nonetheless tended to over-blame instrument agencies and under-blame autonomous agencies, relative to baseline direct action. Recall that Experiment 3 found that, on a scale of 1-10, participants assigned over five units of blame to instrument agencies, even though they possessed no authority to do otherwise. On the other hand, participants slightly underblamed autonomous agencies, by 0.264 on average, although the degree to which the autonomous agencies were underblamed was not significant on a paired \emph{t} test. So, while participants identified differences in responsibility \emph{among} agents, they generally failed to do so \emph{between} principals and agents. 

The difficulty, it seems, is related to assigning blame in those cases where an agency possesses something less than complete authority over a policy. The degree to which voters misattribute blame appears to decline as the agent gains independence from the principal. There are several possible explanations for why instrument agencies are blamed even when they possess little to no efficacy over the resulting policy. In cases in which an actor is linked to an outcome, but plays no clear causal role in producing the outcome, observers may attempt to discern something about the actor's character \citep{Nadler2012}. Cushman et al. hypothesize that, in cases in which actors possess neither causal responsibility for an outcome nor harmful intent, observers might nonetheless find such actors blameworthy if it could be inferred that they possess some attitude toward the harmful outcome \citep{Cushman2012}. For instance, if an individual bets against a company, and thus stands to benefit from a decline in a company's stock price, observers might find the short seller blameworthy, even though he did not cause the decline nor intend it. Likewise, participants here may attribute some affinity, ideological or otherwise, between the instrument agency and the legislature. Such an attribution would not be surprising, as voters might have many reasons to believe that legislatures generally employ agencies that they believe will faithfully act on their behalf.\footnote{Alternatively, participants might have decided that the instrument agency, even though it operated under strong legislative oversight, could, in fact, have acted differently. This is of course true: Individuals possess an ineliminable agency. The hypothetical agency officials carrying out the legislatures' plans could have done otherwise, even if that meant abandoning their post at the agency.} 

Delegation, it seems, has the potential to affect legislative delegation in an even more straightforward way than the clarity thesis envisions. To the question, Why do legislatures delegate power, the clarity thesis responds, They do so in order to create and exploit informational asymmetries in order to fool voters. As Fox and Jordan argue, if certain conditions are met, ``voters cannot perfectly distinguish beneficial delegations from harmful delegations and, as a result, beneficial delegations by well- intentioned politicians provide `cover' for harmful delegations by biased politicians.  \citep[p. 844]{Fox2011} Increased informational asymmetries ``allow[] politicians to shift or obscure responsibility for controversial decisions.'' \citep[p. 145]{Nzelibe2010} Opacity increases incentives for officials to engage in rent-seeking behavior \citep{Mesquita.draft}. The additional complexity created by delegation might allow shirk-prone officials the opportunity to do so. This, in turn, ``may reduce the efficacy of voter discipline, because voters are unable to assign responsibility accurately and effectively.'' \citep[p. 145]{Nzelibe2010} But these findings, which seem to corroborate research on delegation and blame performed in other contexts, suggest a less complicated story. Legislators need not engage in the complicated pattern of behavior described in the clarity thesis; they appear to face less blame for unpopular policy outcomes \emph{simply} by delegating, even the most minimal of authority. More generally put, a regime that allows delegation allows legislators to escape some of the blame they would have faced by enacting policy directly.

\section{Open Questions and Future Research}

Proponents of Derivative Accountability argue that the best way in which to legitimate the administrative state is to give expansive interpretive powers to Executive Branch.\footnote{The effect, then, is that the Executive is (potentially) evaluated along an increased number of policy dimensions. As Gersen observes, ``[c]itizens vote on a bundle of policy positions,'' so we can treat each vote as ``a weighted average of voter approval (or disapproval) on all relevant dimensions.'' \citep[p. 313]{Gersen2010b} Assuming the Executive has some constant set of non-agency dimsensions along which it is evaluated, Derivative Accountability implies that voters add agency-based dimensions to their set of possible evaluative dimensions. And, further, it supposes that the weight attached to the agency-based dimensions will be non-trivial.} \emph{Chevron} and its progeny, the theory runs, accomplish this by maximizing transparency. A ``fundamental precondition of accountability in administration'' then-Professor Elena Kagan argued, is ``the degree to which the public can understand the sources and levers of bureaucratic action.'' \citep[p. 2332]{Kagan2001b} Although agencies are, by their very nature, ``the ultimate black box of government,'' if ``clear lines of command,'' running from the Executive to the agencies, are established, then administrative power can be exercised ``in ways the public can identify and evaluate.'' \citep[p. 2332]{Kagan2001b} In a similar vein, Stephen Holmes argues that separation of powers can ``sort out unclear chains of command, and help overcome a paralyzing confusion of functions.''\citep[p. 165]{Holmes1995} If courts endow the Executive Branch with responsibility over making policy choices and interpretive choices, and the President accepts responsibility for this role, then the electorate will link the two sets of actors. Most importantly, the electorate will evaluate the actions of the agency, then assign those actions to the Executive, as a part of their evaluation of the Executive. The electorate makes a judgment as to the credit or blame of the agency, then assigns that score to the executive. Derivative Accountability, then, holds that transparency is improved, and information is increased, through the process of assignation: the electorate treats the actions of agencies as the actions of the President. In this way, \emph{Chevron} maximizes the responsiveness of policy to majoritarian preferences.

The dynamics presented here could induce the Executive Branch to publicly clarify its role vis-a-vis agencies. Recall that attributions of blame deviated less from baseline cases as agencies acted with more policymaking authority and less legislative oversight. Thus, broad delegations of power to the Executive may \emph{enhance} the performance of agencies by inducing the Executive Branch take extra care of policy outcomes in these situations. By contrast, attributions of blame deviated the most from the baseline scenarios in those cases where agencies exercised less policymaking authority and were more subject to legislative oversight. The error in these cases was to blame agencies \emph{more}, not less. Thus, Derivative Accountability may result in undeserved over-blaming the Executive Branch where Congress has delegated away only minimal authority. In these situations, the Executive Branch might emphasize the relative power of Congress over the agency. 

\section{Conclusion}

This paper provides the first study of delegation and blame to \emph{directly} test third-party observers' beliefs about the blameworthiness of the principals' and agents' behaviors. Rather than rely on indirect measures, the experiments reported here directly measure third-party observers' evaluations of the blameworthiness of the principals' and agents' behavior. It does so in a particularly important context: legislative delegation of political authority to administrative agencies. 

Individuals' ability to accurately blame officials for bad policies is at the heart of the Supreme Court's post-\emph{Chevron} jurisprudence. In \emph{City of Arlington v. F.C.C.}, the Court took up the question of whether a reviewing court should apply \emph{Chevron} to an agency's determination of its own jurisdiction.\footnote{33 S.Ct. 1863 (2013)} Most agree that \emph{Chevron} generally gives agencies, rather than the courts, the authority to interpret ambiguous provisions in statutes that they clearly possess the authority to administer; but what of agencies' authority to decide if they possess the authority to administer the statute? \emph{City of Arlington}, like many of the foundational administrative law decisions that preceded it, answered this question by relying, in part, on \emph{Chevron}'s claim that agencies are better positioned to resolve ``competing policy interests'' than are the courts.\footnote{33 S.Ct. at 1873.} This idea --- that the Legislative and Executive branches together should work out competing policy interests where possible --- is rooted in \emph{Chevron}'s logic of political accountability via the administrative state.   

Political accountability, however, ``hinges on the principals’ [in this case, the voters'] ability to tailor sanctions and rewards to choices made by their agents,'' which, in turn, ``depends ... on the principals’ access to a clear picture of those choices and their relationship to policy outcomes.'' \citep[p. 1]{Mesquita.draft} These results suggest that political accountability is not independent of institutional design. The findings reported here, however, suggest that the very institutional arrangement that brought about the need for derivative liability may end up biasing it. 

\newpage

\emph{Appendix A: Paired T-Tests for Instrument Conditions in Experiment 3}

\begin{center}
    \begin{tabular}{ | l | l | l | p{3cm} |}
    \hline
    Scenario & Direct - Blame & Indirect - Blame & p-value \\ \hline
    Transmission & 8.765385 & 4.391026  & $<$ 2.2e-16 \\ \hline
    Implementation & 8.347436 &  4.869231 & 2.505e-14 \\ \hline
    Creation & 8.638462 &   6.561538 & 6.29e-08 \\
    \hline
    \end{tabular}
\end{center}

\newpage

\bibliographystyle{plainnat}
\bibliography{Clarity.11.10.14.JELS}
\end{document}
